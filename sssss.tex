\documentclass{scrartcl}

\usepackage[T1]{fontenc}
\usepackage{iftex}
\iftutex 
        \usepackage{fontspec}
        \setmainfont{TeX Gyre Schola}
        \setsansfont{TeX Gyre Heros}
        \setmonofont{TeX Gyre Cursor}
        \usepackage{unicode-math}
        \setmathfont{TeX Gyre Schola}
\else
        \usepackage{tgschola}
        \usepackage{tgheros}
        \usepackage{tgcursor}
\fi

\title{A brief essay on life, living, resonsibility, and suicide.}
\author{Runxi Yu}
\date{\today}

\begin{document}
\maketitle

There is something that felt fundamentally wrong when I learned about
social contract theory. ``How can it be,'' I ask, ``that an individual
must maintain their responsibility in the so-called social contract, if
such a contract is not even consensual?'' In fact, the social contract
falls short of any of the three pillars of contracts: It is not
informed, for it is impossible to predict the experience of living
without first coming to life; it is not voluntary, for there is no
choice but to exist and live; and it is not consensual, for the same
reason. Some may then proceed to ask: ``then what is it, that mandates
living, if it could not be adequately modelled as a contract?'' But this
question presumes that the reason to be found must imply that there is a
mandate to live. I shall not assume this.

What naturally results from this line of thought, is the hypothesis that
people have the right to an option of voluntary death, for if we are
unable to make the choice before birth, we ought to, as an imperfect
remedy, be able to defer this choice to when we are conscious and
somewhat informed. (For the sake of brevity, let us not consider those
who are unable to make such a choice.) I shall explore objections by
establishing a thought experiment, as suggested by Mr.\ Naylor: imagine
a world where there are ``suicide booths'' every here and there, and let
us assume that it is impossible to be involuntarily subject to one, that
there is no afterlife, that the experience of suicide is akin to
permanently falling asleep, that the experience is painless, and that
the remains of any body simply disappears.

A few concerns thereby arise. What if a person is temporarily
overwelmed? There are a plethora of conditions which may lead an
individual to consider death. Perhaps someone lost all of their savings
due to a bankruptcy. Perhaps a student failed an exam. Perhaps someone
just broke up with their romantic partner. We seem to be concerned, that
people are prone to impulses that they would regret; so what if,
hypothetically, we also implement a system that requires the person to
have the consistent thought of suicide for a set duration, or another
criteria of a similar nature.  But then, aside from the infeasibility of
agreeing on such a standard, this is a step in the way of people's
exercise of their supposedly fundamental right. And it may perhaps seem
arrogant that we assume that temporary pains are insufficient
justifications to suicide; there are many decisions that people could
\emph{actually} regret that are often made out of an impulse, and
suicide isn't even one of them as it is impossible to regret it after it
has been completed. In any case, it seems wrong to impose these
conditions.

It might be beneficial to analyze why we fear the hypothetical suicide
booths. Although this varies from person to person, I believe that it
boils down to (1) the fear that I myself would lose potential freedom of
choice in the future that could lead me to a desirable life, and (2) the
fear that I could lose people I care for because of decisions on suicide
taken by themselves on their own behalf.

\end{document}
% vim: tw=72 colorcolumn=73

