\documentclass{scrartcl}

\usepackage[T1]{fontenc}
\usepackage{iftex}
\iftutex 
        \usepackage{fontspec}
        \setmainfont{TeX Gyre Pagella}
        \setsansfont{TeX Gyre Heros}
        \usepackage{unicode-math}
        \setmathfont{TeX Gyre Pagella}
\else
        \usepackage{tgpagella}
        \usepackage{tgheros}
\fi

\title{A brief memoir on life, living, resonsibility, and suicide}
\author{Runxi Yu}
\date{}

\begin{document}
\maketitle

There is something that felt fundamentally wrong when I learned about
social contract theory. How can it be, that an individual must maintain
their responsibility in the so-called social contract, if such a
contract is not even consensual? In fact, the social contract falls
short of any of the three pillars of contracts: It is not informed, for
it is impossible to predict the experience of living without first
coming to life; it is not voluntary, for there is no choice but to exist
and live; and it is not consensual, for the same reason. Some may then
proceed to ask: ``then what is it, that mandates living, if it could not
be adequately modelled as a contract?'' But this question presumes that
the reason to be found must imply that there is a mandate to live. I
shall not assume this.

What naturally results from this line of thought, is the hypothesis that
people have the right to an option of voluntary death, for if we are
unable to make the choice before birth, we ought to, as an imperfect
remedy, be able to defer this choice to when we are conscious and
somewhat informed. (For the sake of brevity, let us not consider those
who are unable to make such a choice.) I shall explore objections by
establishing a thought experiment: imagine a world where there are
``suicide booths'' every here and there, and let us assume that it is
impossible to be involuntarily subject to one, that there is no
afterlife, that the experience of suicide is akin to permanently falling
asleep, that the experience is painless, and that the remains of any
body simply disappears.
% This was suggested by Mr. Naylor but it'd be very random to just
% mention people in here...

A few concerns thereby arise. What if a person is temporarily
overwelmed? There are a plethora of conditions which may lead an
individual to consider death. Perhaps someone lost all of their savings
due to a bankruptcy. Perhaps a student failed an exam. Perhaps someone
just broke up with their romantic partner. We seem to be concerned, that
people are prone to impulses that they would regret; so what if,
hypothetically, we also implement a system that requires the person to
have the consistent thought of suicide for a set duration, or another
criteria of a similar nature.  But then, aside from the infeasibility of
agreeing on such a standard, this is a step in the way of people's
exercise of their supposedly fundamental right. And it may perhaps seem
arrogant that we assume that temporary pains are insufficient
justifications to suicide; there are many decisions that people could
\emph{actually} regret that are often made out of an impulse, and
suicide isn't even one of them as it is impossible to regret it after it
has been completed. In any case, it seems wrong to impose these
conditions.

It may be helpful then to analyze why I otherwise fear the hypothetical
suicide booths. I believe that it boils down to (1) the fear that I
myself would lose potential freedom of choice in the future that could
lead me to a desirable life by taking the option of suicide, and (2) the
fear that I could lose people I care about because of decisions on
suicide taken by themselves on their own behalf.

The latter is, by my standard, unsound, at least as a moral argument
against the act of suicide. I have had no agreement with anyone,
formal or otherwise, that obligated their continued living; the
expectation otherwise arises from the social context of suicide being
considered taboo, and mere societal distaste is not remotely
sufficient as a reason towards this issue of its kind. Then there is
grevience; but while the minimization of others' grevience and misery is
certainlly an admirable goal, it is still an insufficient reason in
theory, as one's first and foremost responsibility is to themself. While
the grevience of others may be the strongest argument against suicide on
emotional terms, the socially-established nature of the assumptions it
depends on voids it any possibility of becoming a responsibility per se.
% What about living to fulfill other contractual obligations?
% What about just... fear?

The former is more complicated. A related argument quite commonly seen
as one against suicide is that such an exercise of body autonomy denies
the very liberty that made the act possible in the first place; it
negates the presupposed condition. But this is not at all unique to the
question of suicide, and is applicable to all agreements that trade
some liberty for another where that liberty is scarcely available.
However it is also to be recognized that there is a critical difference
between those and suicide, as the latter leads to a complete abscence of
choice in the future, to regret suicide or otherwise. It deprives the
subject of all liberty and of all choice whatsoever. Could the
succession of life itself void the necessity of liberty altogether?
While I find it tempting to say yes, doing so would provide
justification for many types of non-consensual and artificial
deprivation of life too. I believe that it boils down to entrusting that
the subject has evaluated their choices and concluded that the cecession
of life is the more ideal than continuing to endure what they have been
enduring. And for the fact that only the subject themself have any
possibility of obtaining sufficient information to make this judgement,
it is only appropriate that we leave this to themself.

[I don't really want to end this on such a note. I would probably come
back and re-edit this.]

\end{document}
% vim: tw=72 linebreak
% no colorcolumn=73

