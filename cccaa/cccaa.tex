\documentclass[ignorenonframetext, aspectratio=169]{beamer}
%\documentclass[parskip=half-]{scrartcl}
%\usepackage{beamerarticle}

\usetheme{Hannover}

\usepackage{tgschola}
\usepackage{tgheros}
% \usepackage{tgcursor}
\usefonttheme{professionalfonts}
\renewcommand{\familydefault}{\rmdefault}
\usepackage{microtype}

\usepackage[notes, backend=biber, useibid=true]{biblatex-chicago} % remove noibid when fixed capitalization
\addbibresource{/home/runxiyu/phil/db.bib}
\newcommand{\nopage}{\_\_\_\_}
\newrobustcmd*{\Runcite}{\bibsentence\runcite}

\usepackage{graphicx}

\usepackage{hyperref}
\hypersetup{
	colorlinks=false,
}

\only<presentation>{
	\setlength{\parskip}{1ex}
}

\title{Affirmative Action, Again}
\subtitle{\textit{Students for Fair Admissions \textup{v.}\ Harvard/UNC} (2023)}
\author{Runxi Yu}
\date{August 10, 2024}

\begin{document}
\only<article>{
	\maketitle
}

\only<presentation>{
	\begin{frame}
		\titlepage
	\end{frame}
}

\only<article>{
	\textbf{Note:}
	You are viewing this presentation in article mode.
}

\begin{frame}[fragile]{Miscellaneous information}
\begin{itemize}
	\item Except for text included within double quotation marks and text within a \verb|quote|/\verb|quotation| environment, all text in this slide show is in the public domain, available under Creative Common Zero version 1.0, or the 0-clause BSD license, at your option.
	\item This does not mean that the entire slide show (especially images) is public domain. However, quotes of court opinions are public domain; see 17 U.\,S.\,C.\ {\S} 105 and \Runcite{banks}.
	\item The source code of this slide show is available in \href{https://git.sr.ht/~runxiyu/philo}{\texttt{https://git.sr.ht/\~{}runxiyu/philo}}. I might send tarballs to the WeChat group.
	\item This slide show uses \texttt{biblatex-chicago} for citations, which are generally of the form ``\textit{Petitioner} v.\ \textit{Respondent}, Volume U.\,S. Start-page, Page (Year)''.
\end{itemize}
\end{frame}

\section{Background}

\begin{frame}{What is SFFA \textit{v.}\ Harvard?}
\begin{itemize}
	\item Harvard and UNC used to have affirmative action based on race in their admissions process.
	\item SCOTUS ruled that's not okay. \Runcite{harvard}.
\end{itemize}
\end{frame}

\begin{frame}{Key question}
\begin{center}\bfseries
	Is racial affirmative action a form of undue discrimination?
\end{center}
\end{frame}


\section{Statutes}

\begin{frame}{Relevant clauses in the 14th Amendment}
\begin{block}{Due Process Clause}
	``\ldots any State [shall not] deprive any person of life, liberty, or property, without due process of law.''
\end{block}
\begin{block}{Equal Protection Clause}
	``All persons born or naturalized in the United States, and subject to the jurisdiction thereof, are citizens of the United States and of the State wherein they reside. No State shall make or enforce any law which shall abridge the privileges or immunities of citizens of the United States; nor shall any State deprive any person of life, liberty, or property, without due process of law; nor deny to any person within its jurisdiction the equal protection of the laws.''
\end{block}
\end{frame}

\begin{frame}{Relevant clauses in the Civil Rights Act}
\begin{block}{42 U.\,S.\,C.\ \S\ 2000d}
	``No person in the United States shall, on the ground of race, color, or national origin, be excluded from participation in, be denied the benefits of, or be subjected to discrimination under any program or activity receiving Federal financial assistance.''
\end{block}
\end{frame}

\section{\citetitle{bakke}}

\begin{frame}{Reasons for Affirmative Action?}
\Runcite{bakke}
\begin{enumerate}
	\item[(i)] ``reducing the historic deficit of traditionally disfavored minorities in [...] schools''
	\item[(ii)] ``countering the effects of societal discrimination''
	\item[(iii)] ``obtaining the educational benefits that flow from an ethnically diverse student body''
\end{enumerate}
\end{frame}
% \appendix

\begin{frame}{Justice Powell's objections in \citetitle{bakke}}
Justice Powell observed that
\begin{enumerate}
	\item[(i)] is insufficient because it was akin to ``[p]referring members of any one group for no reason other than race or ethnic origin'', the very definition of racial discrimmination
	\item[(ii)] was insufficient because it was ``an amorphous concept of injury that may be ageless in its reach into the past'', \runcite[307]{bakke}. But Justice Powell recognized diversity as a compelling interest: ``[A] black student can usually bring something that a white person cannot offer.'' \Runcite[316]{bakke}.
\end{enumerate}
\end{frame}

\section{Academic diversity}

\begin{frame}{Academic diversity}
``Admissions programs fail to articulate a meaningful connection between the means they employ and the goals they pursue.'' \Runcite[\nopage{} (slip op.\ 24)]{harvard}. The court thinks that there is no meaningful connection between racial affirmative action to academically-significant diversity.

The court objects:
\begin{itemize}
	\item Vague classifications (e.g. ``Asians'' is a vague category)
	\item The ``belief that minority students always (or even consistently) express some characteristic minority viewpoint on any issue.''
\end{itemize}
Are there specific academic benefits that only racial diversity could bring?
\end{frame}

\section{Economic status}

\begin{frame}{Economic status}
Some colleges, including Harvard, claim that being in a race that is historically disadvantaged means that the student is more likely to be in a poorer socio-economic status, and improved education is significant in creating equality for the families of these students.

Here, race is being used as a proxy for socio-economic status. I believe that race is not a particularly good proxy, not only because of its wide-ranging legal issues that would not be raised if socio-economic status is considered alone, but also because of its alleged ineffectiveness evident in \citetitle{harvard}. ``While Harvard professes interest in socioeconomic diversity, for example, SFFA points to trial testimony that there are 23 times as many rich kids on campus as poor kids.''  \Runcite[\nopage{} (\textsc{Gorsuch}, J., concurring) (slip op., at 13)]{harvard} (internal quotation marks omitted).
\end{frame}

\section{Seggregation}

\begin{frame}{Seggregation}
``Passively eliminating race classifications did not suffice when \textit{de facto} seggregation persisted.'' \Runcite[\nopage{} (\textsc{Sotomayor}, J., dissenting) (slip op., at 12)]{harvard} (quoting \runcite[440–442]{green}). The court previously has also stated that a school board ``had to do more than abandon its discriminatory purpose'', it ``had an affirmative responsibility to integrate.'' \Runcite[538]{brinkman}.

Rather than just saying that affirmative action could be used to \emph{correct past wrongs}, perhaps out of (questionable notions of) collective responsibility, these are examples of arguments that affirmative action is necessary in furthering integration and countering societal discrimination.

But of course, we live in a world that is vastly different from the world that existed during \citetitle{brown} and \citetitle{green}. I have not gone through the entirety of the Joint Appendices for \citetitle{harvard} and similar cases, but the prevalence of anything that could be considered ``segregation'' in society should be at a minimum, or at least on an entirely different level than that of the 1970s.

Is there still a compelling interest in increasing integration?
\end{frame}

\section{End}
\begin{frame}{End}
``[A]ll race-conscious admissions programs [must] have a termination point''; they ``must have a logical end point''; their ``deviation from the norm of equal treatment'' must be ``a temporary matter''; ``[e]nshrining a permanent justification for racial preferences would offend this fundamental equal protection principle.'' \Runcite[342–343 (internal quotation marks omitted)]{grutter}.

Affirmative action is a policy that adapts to changing social conditions. When should affirmative action end?
\end{frame}

\section{Autonomy}
\begin{frame}{Autonomy}
``Admission is not an honor bestowed to reward superior merit or virtue. Neither the student with high test scores nor the student who comes from a disadvantaged minority group morally deserves to be admitted. Her Admission is justified insofar as it contributes to the social purpose the university serves, not because it rewards the student for her merit or virtue, independently defined.'' \Runcite[174]{justice}.

To what extent do responsibilities of the state apply to universities? To  what extent should they have autonomy over admissions decisions?
\end{frame}

\end{document}
