\documentclass{scrartcl}

\usepackage[T1]{fontenc}
\usepackage{iftex}
\iftutex 
        \usepackage{fontspec}
        \setmainfont{TeX Gyre Schola}
        \setsansfont{TeX Gyre Heros}
        \setmonofont{TeX Gyre Cursor}
        \usepackage{unicode-math}
        \setmathfont{TeX Gyre Schola}
\else
        \usepackage{tgschola}
        \usepackage{tgheros}
        \usepackage{tgcursor}
\fi

\title{The Null Hypothesis}
\author{Runxi Yu}
\date{\today}

\begin{document}
\maketitle

The consideration of any statement comes with a null hypothesis.

\end{document}
% vim: tw=72 colorcolumn=73

% Wikipedia: In neither case is the null hypothesis or its alternative proven; the null hypothesis is tested with data and a decision is made based on how likely or unlikely the data are. This is analogous to the legal principle of presumption of innocence, in which a suspect or defendant is assumed to be innocent (null is not rejected) until proven guilty (null is rejected) beyond a reasonable doubt (to a statistically significant degree). 
% I don't think this makes sense.
