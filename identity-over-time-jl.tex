\documentclass[a4paper,english,12pt]{scrartcl}
\usepackage[unicode=true, bookmarks=true,bookmarksnumbered=false,bookmarksopen=false, breaklinks=false,pdfborder={0 0 0},pdfborderstyle={},backref=false,colorlinks=false]{hyperref}
\hypersetup{pdftitle={In what sense are you the same person today that you were when you were ten?},
 pdfauthor={Runxi Yu},
 pdfsubject={2023 John Locke Essay Competition, Philosophy Question 2}}
\pdfpageheight\paperheight
\pdfpagewidth\paperwidth
\usepackage{newtxmath}
\usepackage{newtxtext}
\usepackage[style=apa]{biblatex}
\addbibresource{db.bib}
\usepackage{microtype}

\title{In what sense are you the same person today that you were when you were ten?}
\author{\href{https://www.andrewyu.org/}{Runxi Yu}}
\date{For the 2023 John Locke Essay Competition, Philosophy Question 2}

\begin{document}
\maketitle

\noindent When the Ship of Theseus has its all parts replaced one
after another until no original components remain, is it still the
Ship of Theseus, or is it a new ship altogether?  Similarly, when
most of my body cells are constantly renewed, coupled with my mindset
being continuously transformed by new information and acquaintances,
am I still the same person as the ten-year-old ``me''?

This essay explores the continued \emph{personal} identity from two
perspectives. First is the concept of \emph{individuality}, i.e. whether
there is a delimited, discrete, and cohesive existence of ``me''
in the first place. An affirmative answer positing the delineated
boundary of individual existence is the prerequisite to the second
question of \emph{uniqueness}, which makes each individual different
from others. Two layers of analysis are applied to uniqueness
--- the \emph{identification} of the uniqueness of personal identity
at a set time compared with other individual entities, and the \emph{continuity}
of the person's uniqueness over time.

\section{The Individuality of ``Me''}

Aristotelianism portrays the ``individual'' as a distinct substance,
differentiated from accidents such as qualities, quantities, or relations
\parencite{MetaphysicsZeta}. The substance of a human is the body.
Each organ and system have their own functions, yet they orchestrate
to keep us alive. Unlike coral reef, consisting of indistinct strands
of polyps, the human body has a generally clear physical boundary
--- the skin. Therefore, our body is a discrete individual
entity.

The concept of individuality expands beyond the biological dimension.
\textcite{LockeHumanUnderstanding} ascribes personal identity to
the continuity of consciousness, hinging on memory. \textcite{DavidHume} sees the self
as a ever-changing “bundle” of perceptions or experience. \textcite{Mead1934MindSA} defines identity by social relations,
which change over time as we forge new connections and networks.

Our memory, experience and social relations, which are applied to
define our identity by the above philosophers, can't exist independently
but are rather supported by the substance of our body, through which we interact
with the world. Our individuality is therefore a unity of plurality,
which comprises of biological, empirical and social parameters. The analysis of individuality
on all its parameters is beyond the length of this essay. But since
they are dependents of our physical existence that are clearly delineated,
this essay will move from discussing discrete individuality of our
substance to the exploration of the uniqueness of each individual,
and the possible continuity of this uniqueness.

\section{The Uniqueness of ``Me'' and its Continuity}

Canned tuna on shelves are individual identities but they are 
practically not unique to the consumer. As long as they are still
in the shelf life, no one cares to take one can instead of another.
There is nothing particular about an industrial product
that makes it stand out from others mass produced or assembled at
the same factory.

Each \emph{person}, however, is unique in their biological identity,
empirical identity, and social identity. This essay explores people's
unique characteristics through these three lens, and how they are
continued with the passage of time. I hereby propose that the uniqueness
of a person is preserved over time if the person's unique characteristics
are preserved over time.

\subsection{Biological Identity and Its Continuity}

``On average, the cells in your body are replaced every 7 to 10 years''
\parencite{what-cells}. Despite the tempting but inaccurate interpretation
that human bodies are renewed every decade or so, there is significant
constancy inside our body. For example, most neurons do not regenerate.
But even if all cells do, the regeneration of our body's \emph{constituent
parts} does not entail that our biological \emph{properties} change
over time. Many biological properties that are perceived to be identifying
for individuals, such as DNA sequence and fingerprints, generally
stay constant throughout our life. The wide application of fingerprints
in identifying people in user authentication and criminal investigations
suggests that fingerprints are unique and enduring.

The uniqueness of our biological system and the continuity of the
uniqueness are reinforced by the ``Self/Non-self'' theory by virologist
Sir Frank Burnet \parencite{BurnetFenner}. The distinction of ``self''
from most other entities elicits defensive responses against pathogens,
the tolerance of a graft from itself, and the rejection against grafts
from a donor organism in transplantation with few exceptions such
as isografts (grafts between identical twins) \parencite{pradeu2011limits}.
Adaptive immunological memory is formed when specialized memory lymphocytes
are produced, which would trigger a more rapid and effective immune
response on the next infection by a pathogen with similar antigens.
Adaptive immunity leaves a long-term mark on our body; while COVID
immunity lasts about 6 months, smallpox immunity lasts for decades
\parencite{Taub2008-ps}. Therefore, my special immunity spans over
time due to the existence of immunological memory.

In conclusion, although most cells in our body regenerate constantly,
our biological identity does not change since it consists not in the
collection of components but in the properties that emerge therefrom.
The fact that our identity makers such as our DNA sequence, fingerprints,
and immunity transcend over time warrants the conclusion that our
biological identity spans diachronically.

\subsection{Empirical identity and Its Continuity: Memory}

\textcite{LockeHumanUnderstanding} postulates the memory theory of
identity. A person's identity is tied to their memory, i.e. their
ability to recognize their past experiences as their own, and to connect
them to their present consciousness. Endel Tulving further analyzes
the concept of memory by distinguishing between procedural memory
(skills, e.g. muscle memory), semantic memory (factual information)
and episodic memory (personal experiences) \parencite{OutOfThePast}. 

Semantic memory as a whole rather than individual piece of factual
information contributes to identity. Many people can retrieve the
knowledge that Napoleon was defeated at Waterloo, and this does not
grant them identity because this piece of knowledge lacks uniqueness
that tells one person apart from another. However the assembly of
semantic memory of each person is still an idiosyncratic feature,
since one person may share some semantic memories with people of similar
educational background but it is unlikely that the entire knowledge
system of each person is identical. Though semantic memories are constantly
renewed by newly accumulated knowledge or attrition as time goes by,
the changing process is gradual. Moreover, the cognitive structure
underlying the acquired information, the critical thinking cultivated
by the analysis of those information, and the mindset of each individual
are relatively stable, as well as uniquely forged by each person's
idiosyncratic assembly of semantic memories.

Locke's theory appeals to episodic memories as a necessary condition
for uniqueness of individuals as well as sameness over time. ``As
far as this consciousness can be extended backwards to any past Action
or Thought, so far reaches the Identity of that Person; it is the
same self now it was then; and 'tis by the same self with this present
one that now reflects on it, that that Action was done'' (Book II.xxvii.9).

This theory might be challenged from two perspectives. Firstly, multiple
people could have gone through the same events, and the shared experience
lacks uniqueness to individuals, hence absence of identity since it
is based on the premise of uniqueness as previously explained. However,
although different people may share the same episodes, they were different
agents in those moments and engaged from their own angles respectively.
Therefore, one's memory of these episodes is still distinct from that
of others'.

The second and more common challenge to Locke's memory theory is that
I may remember episodes of five years ago, and me of five years ago
can recall events of ten years ago, but the current me cannot recall
most of the life of ten years ago when I was a ignorant and happy
kindergartner. Am I still the same person of ten years ago? I would
tackle this question by the transitivity principle: if $a=b$ and
$b=c$, then $a=c$. If I share the identity of me from five years
ago, and me of five years ago was still the same person as the one
of ten years ago, I and the one of ten years ago still have the same
identity on the principle of transitivity.

An exception would be rare cases of memory loss due to pathological
conditions in the brain. The loss of memory is a rupture in one's
life. If that loss is significant enough to affect one's personality,
I argue that they are not the same person.

Procedural memories are typically diachronically enduring --- if
we have learned how to perform a skill in our childhood, the procedural
memory is carried on throughout our life. For example, even if we
have not touched bikes for ten years, once we get onto one, we can
still ride on like we have not stopped riding all these years. As
in the case of semantic memory, the assembly of procedural memories
are also idiosyncratic to each individual. These unique procedural
memories stay, without us being aware of it.

Therefore, the above analysis of semantic, episodic and procedural
memories suggests that what makes people unique and also identical
with themselves along the temporal axis is their remembering or being
able to remember the knowledge assembly, the episodes to which they
were witness or agent, and also the skills they acquire and stay.
Personal identity consists in memory.

\subsection{Social identity}

Our identity is not only biologically and empirically
determined, it is also socially constructed.

According to \textcite{Mead1934MindSA}, the self emerges from social
interactions. He divides the self into two components: the ``me''
and the ``I''. The ``me'' represents the organized set of attitudes
of others that the individual assumes. It's the social self, the part
of us that is formed through interaction with others and with the
social environment. It embodies the expectations and norms of the
community, allowing us to predict how others will react to us. The
``I'' is the immediate response of an individual to others. It is
the spontaneous, unpredictable, and creative part of the self. The
``I'' reacts to the ``me'', and it's through this dialogue that
we create meaning, make decisions, and ultimately take action. So,
for Mead, identity is continually created and recreated through the
social interactions we have, the roles we take on, and the dialogue
between the ``I'' and the ``me''. Since social interactions are
dynamic, our identity is fluid rather than fixed.

However, the existence of the ``me'', which embodies the attitudes,
roles, and rules assimilated from the social environment, provides
a consistent framework that guides behavior and thought. It ensures
the stability and continuity of our identity. Meanwhile, the elasticity
of social interactions allows us to grow inside the parameters of
the continuity of identity.

The impact of this continuity is huge. Its absence would translate
into lack of accountability for our own behavior and decision-making,
since we would not bear responsibility for our mistakes or crimes
if we could legitimately renew our identity every few years or
months. Property rights would be bygone and personal relationships chaotic
since filial, marital relationships vanished together with the continuity
of social identity.

\section{Conclusion}

My identity emerges from my unique biological, empirical and social
properties. In my case, all three of these properties have continuously
developed since I was ten, but always with some constancy, be it the
preservation of the uniquely identifying biological characteristics,
my unique memory, or my our
social identity shaped through social interactions. I maintain my
sense of self through time due to the constancy in these properties.
I am therefore intrinsically linked to the ten-year-old child I was,
biologically, empirically, and socially.

\printbibliography

\end{document}
